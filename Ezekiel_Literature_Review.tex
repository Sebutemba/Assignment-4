\documentclass[]{report}


% Title Page
\title{}
\author{KIZITO SEBUTEMBA EZEKIEL	213012200		13/U/6970/EVE}


\begin{document}
\maketitle LITERATURE REVIEW ABOUT GOOGLE DRIVE


\begin{flushleft}
	\underline{Introduction}\linebreak
	
	There are a number of applications designed and developed by a one famous tech company, Google. Such applications have seen this world adapt and easily overcome some of the would-be-then challenges with regards to issues like communication, storage, capturing lifetime memories, etc. It’s at this peak that the world has grown to see life-changing ideas in many forms. The main focus here; lies under software.\linebreak
	 A great scholar once said\cite{1} that, software is simply a change of hard to soft. It’s through these ideas that each day that goes by, there’s birth to a new tech company. Among those is  Google which has developed many softwares under the umbrella of applications\cite{2}.\linebreak
	Some of the famous softwares designed by this giant tech company include Google search \cite{3}, a web search engine; Google books\cite{4} ,a search engine for full text of printed books and many more. These can be termed as services. However, there has also been software designed in form of products; call it application software\cite{5}. Such include (Google products specifically); Google hangouts, Google calender, Google photos\cite{6}, etc. This entails the main focus of the literature review about Google drive a one application designed by Google.\linebreak
	
	\underline{Background Of Google Drive}\\
	Google drive is a product designed and developed by Google purposely for file storage\cite{7}and synchronization service\cite{8} . Dating back on April 24, 2012, the app is designed in such a way that it enables users to store files on the Google servers, be in position to synchronize files across devices as well as share files.\\
	In addition to a website, Google Drive offers apps with off line capabilities for operating systems platforms such as Windows and macOS, and Android and iOS with regards to computers, smartphones and tablets respectively. The application software is designed in a way that comes with more functionalities such as Google Docs, Sheets and Slides, and office suite\cite{9}.\\ These permit collaborative editing of documents, spreadsheets, presentations, drawings, forms, and more\cite{10}. These edited and created files are saved in Google Drive through use of office suite.
	Google Drive offers users 15 gigabytes of free storage, with 100 gigabytes, 1 Tera byte, 2 Tera byte, 10 Tera bytes, 20 Tera bytes and 30 Tera bytes offered through optional paid plans. It’s noted that the files uploaded on this platform can be up to 5 Tera bytes in terms of size.\cite{10}\linebreak
	
\underline{	Overview}\\
	Google drive is characterized by two terms; ie Google and drive. The term Google stands “search for information about something somebody using the Internet through a web search engine called Google” where as drive simple stands for “propel or carry along by force in a specified direction. A combination of the two terms stands for “a cloud storage service developed by Google. \\The application software works in the sense that users can change privacy settings for individual files and folders, including enabling sharing with other users or making content public. On the website, users can search for an image by describing its visuals, and use natural language to find specific files, such as "find my budget spreadsheet from last December".\linebreak
	
	This app is sub-categorized under website and Android app. These both offer a backups section to see what Android devices have data backed up to the service as well as a completely overhauled computer application that was released earlier in July 2017. It also allows for backing up of specific folders on the user's computer. It is characterized by a Quick-Access feature which can smartly predict the files needed by a user. The app is also designed to offer unlimited storage, advanced file audit reporting, enhanced administration controls, and also greater collaboration tools for teams.
	It is said that as of March 2017, Google Drive has over 800 million active users, and as of September 2015, it had over one million organizational paying users. The application is also said that as of May 2017, there are over two trillion files stored on the service.
	
	 \bibliographystyle{unsrt}
	\bibliography{ref}
	
	
	
	
	
	
	
	
	
	
	
	
	
	
	
	

\end{flushleft}
\end{document}          
